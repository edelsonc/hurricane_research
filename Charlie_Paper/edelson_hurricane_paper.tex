\documentclass{article}

\usepackage[margin=1in]{geometry}

\title{Hurricane Characterization with Common Inferential Statistics and Machine Learning Techniques}
\author{Charlie Edelson}

\providecommand{\keywords}[1]{\small{\textit{Keywords:} #1}}


\begin{document}
	
	\maketitle

	\begin{abstract}
	PUT ABSTRACT HERE!!!!!! This is where the abstract will eventually go. However, currently we only have a place holder.
	\end{abstract}
	
	\keywords{PUT KEY WORDS HERE!!!!, these, are, important, words}

	\section{Introduction}
	UPDATE BACKGROUND HERE!!!! In this section I will talk about hurricanes and how they're very interesting model systems. Additionally, I'll go on to talk about a little bit of the recent history into modeling them and why people use them.
	
	The purpose of this paper is to investigate the properties of named storms in the Atlantic ocean basin using inferential statistics and machine learning techniques. Quadratic and Cubic polynomial regressions will be used to parameterize and model storm windspeed as a function of time, with coefficients binned across storms. ARIMA models will then be fit to each storm, and the most common models will be further investigated. Additionally, the average number of storms per year, $\lambda$, will be investigated using Bayesian statistics to get a distribution on the $\lambda$. Finally, points in windspeed vs pressure phase space will be clustered using two classical machine learning technique, k means clustering and hierarchical clustering. These results will serve as a starting point for further investigation into storm characterization with statistical methods.

	\section{Methods}
	Unisys 2000-2010 Hurricane/Tropical storm data\cite{Unisys} was used throughout this paper. This data consists of timestamped measurements of storm pressure, temperature, windspeed, latitude, and longitude for all major tropical depressions, tropical storms, and hurricanes between 2000 and 2010. Measurements were taken at approximately six hour intervals. Futhermore, storm name is included for all tropical storms and hurricanes.


	\bibliography{sources}
	\bibliographystyle{ieeetr}

\end{document}